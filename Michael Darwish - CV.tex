%% The MIT License (MIT)
%%
%% Copyright (c) 2015 Daniil Belyakov
%%
%% Permission is hereby granted, free of charge, to any person obtaining a copy
%% of this software and associated documentation files (the "Software"), to deal
%% in the Software without restriction, including without limitation the rights
%% to use, copy, modify, merge, publish, distribute, sublicense, and/or sell
%% copies of the Software, and to permit persons to whom the Software is
%% furnished to do so, subject to the following conditions:
%%
%% The above copyright notice and this permission notice shall be included in all
%% copies or substantial portions of the Software.
%%
%% THE SOFTWARE IS PROVIDED "AS IS", WITHOUT WARRANTY OF ANY KIND, EXPRESS OR
%% IMPLIED, INCLUDING BUT NOT LIMITED TO THE WARRANTIES OF MERCHANTABILITY,
%% FITNESS FOR A PARTICULAR PURPOSE AND NONINFRINGEMENT. IN NO EVENT SHALL THE
%% AUTHORS OR COPYRIGHT HOLDERS BE LIABLE FOR ANY CLAIM, DAMAGES OR OTHER
%% LIABILITY, WHETHER IN AN ACTION OF CONTRACT, TORT OR OTHERWISE, ARISING FROM,
%% OUT OF OR IN CONNECTION WITH THE SOFTWARE OR THE USE OR OTHER DEALINGS IN THE
%% SOFTWARE.

% The font could be set to Windows-specific Calibri by using the 'calibri' option
\documentclass[]{cv}

% For mathematical symbols
\usepackage{amsmath}

% Set applicant's personal data for header
%
% There should be a file named "personal.tex" in the same directory that contains
% \name, \address and \contacts.
% For example:
%
% File personal.tex:
% \name{Guy Incognito}
% \address{123 Fake St. \linebreak Seattle, WA}
% \contacts{(555) 123-4567 \linebreak \href{mailto:guy@incognito.com}{guy@incognito.com} \linebreak \href{https://www.incognito.com/}{incognito.com}}
\input{personal.tex}

\begin{document}

	% Print the header
	\makeheader

	% Print the content
	\begin{cvsection}{Employment}
		\begin{cvsubsection}{Senior Developer / Team Lead}{Kaloom}{2018 -- Present}
			\begin{itemize}
				\item Successfully designed and released greenfield UI product on time, using latest technologies and best practices
				\item Responsible for technical architecture, development process, hiring/onboarding, UI/UX and testing
				\item Spearheaded creation of custom CLI tool to improve product usability for all employees and clients
				\item Played a pivotal role in developing cross-team features, requiring cooperation across departments
				\item Supported professional and technical growth of six junior developers through one-on-one mentorship
				\item Served as Scrum Master, encouraging adherence to Agile processes and facilitating efficient team collaboration
			\end{itemize}
			\small{\textbf{Skills:} React, TypeScript, Go, Python, Docker, Kubernetes, Networking, NETCONF, YANG, Distributed systems}
		\end{cvsubsection}

		\begin{cvsubsection}{DevOps Specialist}{Kaloom}{2017 -- 2018}
			\begin{itemize}
				\item Built CI/CD pipeline to test and deliver microservice components, ensuring rapid feedback for every change
				\item Collaborated with teams to integrate build tools and procedures into their development workflows
				\item Owned and developed VM-based testing platform to simulate network topologies, enabling local testing of changes and reducing hardware contention
				\item Designed and implemented component installation process on Kubernetes cluster to achieve repeatable, automated deployments
			\end{itemize}
			\small{\textbf{Skills:} Go, Docker, Kubernetes, Gradle, Jenkins, Bash, Networking, Microservices}
		\end{cvsubsection}

		\begin{cvsubsection}{Lead Developer}{Learning Bird}{2012 -- 2017}
			\begin{itemize}
				\item Led design and development of all aspects of cloud-based web application with successful bi-weekly releases
				\item Oversaw recruitment, management, and mentorship of development team
				\item Implemented modernization of DevOps architecture with fully automated continuous integration
				\item Managed transition of entire cloud system to Docker-based architecture, enabling fast, portable deployments
				\item Supervised design and creation of offline hardware appliance for use in remote, low-connectivity areas
			\end{itemize}
			\small{\textbf{Skills:} JavaScript, HTML, CSS, PHP, SQL, AWS, Jenkins, Linux}
		\end{cvsubsection}
	\end{cvsection}

	\begin{cvsection}{Education}
		\begin{cvsubsection}{Montreal, QC, Canada}{McGill University}{2008 -- 2012}
			\begin{itemize}
				\item Bachelor of Engineering, Computer Engineering
			\end{itemize}
		\end{cvsubsection}
	\end{cvsection}

	\begin{cvsection}{Skills}
		\begin{cvsubsection}{}{}{}
			\begin{itemize}
				\item \textbf{Frontend:} React, TypeScript, JavaScript, HTML, CSS
				\item \textbf{Backend:} Go, Python, PHP, C, C\#, Bash, SQL
				\item \textbf{DevOps:} Docker, Kubernetes, AWS, Linux, Git, Gradle, Jenkins
				\item \textbf{Networking:} L2-L7 networking, NETCONF, YANG
				\item \textbf{Architecture:} Microservices, Distributed systems
			\end{itemize}
		\end{cvsubsection}
	\end{cvsection}

	\begin{cvsection}{Additional Experience and Awards}
		\begin{cvsubsection}{}{}{}
			\begin{itemize}
				\item \textbf{1st place} MasterCard Masters of Code, Montreal -- 2015
				\item \textbf{1st place} Random Hacks of Kindness -- 2012
				\item \textbf{2nd place} Great Canadian Appathon 2 -- 2011
				\item Developed and published eight games in Windows Phone Store
			\end{itemize}
		\end{cvsubsection}
	\end{cvsection}

	\begin{cvsection}{Languages}
		\begin{cvsubsection}{}{}{}
			\begin{itemize}
				\item English (native)
				\item French (fluent)
			\end{itemize}
		\end{cvsubsection}
	\end{cvsection}

\end{document}

