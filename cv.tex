%% The MIT License (MIT)
%%
%% Copyright (c) 2015 Daniil Belyakov
%%
%% Permission is hereby granted, free of charge, to any person obtaining a copy
%% of this software and associated documentation files (the "Software"), to deal
%% in the Software without restriction, including without limitation the rights
%% to use, copy, modify, merge, publish, distribute, sublicense, and/or sell
%% copies of the Software, and to permit persons to whom the Software is
%% furnished to do so, subject to the following conditions:
%%
%% The above copyright notice and this permission notice shall be included in all
%% copies or substantial portions of the Software.
%%
%% THE SOFTWARE IS PROVIDED "AS IS", WITHOUT WARRANTY OF ANY KIND, EXPRESS OR
%% IMPLIED, INCLUDING BUT NOT LIMITED TO THE WARRANTIES OF MERCHANTABILITY,
%% FITNESS FOR A PARTICULAR PURPOSE AND NONINFRINGEMENT. IN NO EVENT SHALL THE
%% AUTHORS OR COPYRIGHT HOLDERS BE LIABLE FOR ANY CLAIM, DAMAGES OR OTHER
%% LIABILITY, WHETHER IN AN ACTION OF CONTRACT, TORT OR OTHERWISE, ARISING FROM,
%% OUT OF OR IN CONNECTION WITH THE SOFTWARE OR THE USE OR OTHER DEALINGS IN THE
%% SOFTWARE.

% The font could be set to Windows-specific Calibri by using the 'calibri' option
\documentclass[]{cv}

% For mathematical symbols
\usepackage{amsmath}

% Set applicant's personal data for header
%
% There should be a file named "personal.tex" in the same directory that contains
% \name, \address and \contacts.
% For example:
%
% File personal.tex:
% \name{Guy Incognito}
% \address{123 Fake St. \linebreak Seattle, WA}
% \contacts{(555) 123-4567 \linebreak \href{mailto:guy@incognito.com}{guy@incognito.com} \linebreak \href{https://www.incognito.com/}{incognito.com}}
\ifdefined\issample
	\input{personal.tex.sample}
	\else
	\input{personal.tex}
\fi

\begin{document}

	% Print the header
	\makeheader

	% Print the content
	\begin{cvsection}{Employment}
		\begin{cvsubsection}{Senior Developer / Team Lead}{Kaloom}{2018 -- Present}
			\begin{itemize}
				\item Led the launch of a new UI product consisting of over 150 pages and forms, built over the span of 18 months on time and on budget, replacing legacy UI
				\item Increased user satisfaction from 35\% to 95\% by iteratively improving design based on feedback from users
				\item Reduced memory usage by over 10x by streamlining technical architecture, enabling the UI to be deployed on resource-constrained hardware
				\item Decreased development costs and improved product reliability by increasing test coverage to 100\%, allowing errors to be detected earlier in the development cycle
				\item Spearheaded creation of custom CLI tool to enhance product usability for all employees and clients
				\item Played a pivotal role in development of 10 cross-team features, requiring cooperation across 7 departments, and working closely with the VP of R\&D
				\item Grew team from 1 to 4 developers, and supported professional and technical growth of each member through one-on-one mentorship
				\item Enabled accurate estimation and fostered team collaboration by serving as Scrum Master and encouraging adherence to Agile processes
			\end{itemize}
			\small{\textbf{Skills:} React, TypeScript, Go, Python, Docker, Kubernetes, Networking, NETCONF, YANG, Distributed systems}
		\end{cvsubsection}

		\begin{cvsubsection}{DevOps Specialist}{Kaloom}{2017 -- 2018}
			\begin{itemize}
				\item Reduced featured delivery time by building a CI/CD pipeline to test and deliver microservice components, ensuring rapid feedback for every change and handling hundreds of daily commits
				\item Collaborated with 5 teams to improve development efficiency by integrating build tools and procedures into their workflows
				\item Owned and developed VM-based testing platform to simulate network topologies, enabling local testing of changes and reducing hardware contention
				\item Designed and implemented component installation process on Kubernetes cluster to achieve repeatable, automated deployments
			\end{itemize}
			\small{\textbf{Skills:} Go, Docker, Kubernetes, Gradle, Jenkins, Bash, Networking, Microservices}
		\end{cvsubsection}

		\begin{cvsubsection}{Lead Developer}{Learning Bird}{2012 -- 2017}
			\begin{itemize}
				\item Led design and development of all aspects of cloud-based web application with successful bi-weekly releases
				\item Oversaw recruitment, management, and mentorship of development team
				\item Implemented modernization of DevOps architecture with fully automated continuous integration
				\item Managed transition of entire cloud system to Docker-based architecture, enabling fast, portable deployments
				\item Supervised design and creation of offline hardware appliance for use in remote, low-connectivity areas
			\end{itemize}
			\small{\textbf{Skills:} JavaScript, HTML, CSS, PHP, SQL, AWS, Jenkins, Linux}
		\end{cvsubsection}
	\end{cvsection}

	\begin{cvsection}{Education}
		\begin{cvsubsection}{Montreal, QC, Canada}{McGill University}{2008 -- 2012}
			\begin{itemize}
				\item Bachelor of Engineering, Computer Engineering
			\end{itemize}
		\end{cvsubsection}
	\end{cvsection}

	\begin{cvsection}{Skills}
		\begin{cvsubsection}{}{}{}
			\begin{itemize}
				\item \textbf{Frontend:} React, TypeScript, JavaScript, HTML, CSS
				\item \textbf{Backend:} Go, Python, PHP, C, C\#, Bash, SQL
				\item \textbf{DevOps:} Docker, Kubernetes, AWS, Linux, Git, Gradle, Jenkins
				\item \textbf{Networking:} L2-L7 networking, NETCONF, YANG
				\item \textbf{Architecture:} Microservices, Distributed systems
			\end{itemize}
		\end{cvsubsection}
	\end{cvsection}

	\begin{cvsection}{Additional Experience and Awards}
		\begin{cvsubsection}{}{}{}
			\begin{itemize}
				\item \textbf{1st place} MasterCard Masters of Code, Montreal (2015) -- Won 1st place out of 17 teams, granting entry into the International Masters of Code hackathon in San Francisco, by building a web app to enable automatic payments for groups of people with shared expenses
				\item \textbf{1st place} Random Hacks of Kindness (2012) -- Won 1st place out of 14 teams by building a game to raise awareness of healthcare issues in developing countries
				\item \textbf{2nd place} Great Canadian Appathon 2 (Sept 2011) -- Won 2nd place out of 125 teams by building a fully functional, archery game for mobile devices
				\item \textbf{3rd place} Great Canadian Appathon (Mar 2011) -- Won 3rd place out of 94 teams by building a space-themed game for mobile devices
				\item Developed and published eight games in the Windows Phone Store, which were downloaded and installed on over 30,000 devices
			\end{itemize}
		\end{cvsubsection}
	\end{cvsection}

	\begin{cvsection}{Languages}
		\begin{cvsubsection}{}{}{}
			\begin{itemize}
				\item English (native)
				\item French (fluent)
			\end{itemize}
		\end{cvsubsection}
	\end{cvsection}

\end{document}

